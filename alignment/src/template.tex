%% Title
%% Template for proto_err reports
%% Phelim Bradley - psb31
%% 

%% Text following a percent sign (%) until the end of line is treated
%% as a comment.

\documentclass[a4paper,12pt]{article}

%%%%%%%%%%%%%%%%%%%%%%%%%%%%%%%%%%%%%%%%%%%%%%%%%%%%%%%%%%%%%%%%%%%%%%
%% This section is called the preamble, where we can specify which
%% latex packages we required.  Most (but not of all) of the packages
%% below should be fairly standard in most latex documents.  The
%% exception is xspace and the new \latex command, which you probably
%% do not need.
%%%%%%%%%%%%%%%%%%%%%%%%%%%%%%%%%%%%%%%%%%%%%%%%%%%%%%%%%%%%%%%%%%%%%%

%% Bibliography style:
\usepackage{mathtools}
\usepackage{mathptmx}           % Use the Times font.
\usepackage{graphicx}           % Needed for including graphics.
\usepackage{url}                % Facility for activating URLs.
\usepackage{xspace,color}
\usepackage{enumerate}
%\usepackage{csvsimple}
\usepackage{caption}
\usepackage{amsmath}
\usepackage{amsfonts}
\usepackage{amssymb}
\usepackage{subfig}
\usepackage{tabularx}
\usepackage{listings}
 \captionsetup{justification=centering,singlelinecheck=on}




%% Set the paper size to be A4, with a 2cm margin 
%% all around the page.
\usepackage[a4paper,margin=2cm]{geometry}

%% Natbib is a popular style for formatting references.
\usepackage{natbib}
%% bibpunct sets the punctuation used for formatting citations.
%\bibpunct{[}{]}{;}{a}{,}{,}

%% This is an example of a new macro that I've created to save me
%% having to type \LaTeX each time.  The xspace command provides space
%% after the word LaTeX where appropriate.
\usepackage{xspace}
\providecommand*{\latex}{\LaTeX\xspace}

%Make some shortcuts

\def\be{\begin{equation}}
\def\ee{\end{equation}}

\def\bea{\begin{eqnarray}}
\def\eea{\end{eqnarray}}


\def\toprule{\hline}
\def\midrule{\hline}
\def\bottomrule{\hline}
%%%%%%%%%%%%%%%%%%%%%%%%%%%%%%%%%%%%%%%%%%%%%%%%%%%%%%%%%%%%%%%%%%%%%%
%% Start of the document.
%%%%%%%%%%%%%%%%%%%%%%%%%%%%%%%%%%%%%%%%%%%%%%%%%%%%%%%%%%%%%%%%%%%%%%

\begin{document}




\author{Proto\_err}
\date{\today}
\title{  Alignment Report {{ID}}   }
\maketitle

Number of Bases: {{numBases}} \\
Number of Aligned Bases : {{numAlignedBases}} \\
Number of MatchMissMatch Bases : {{M}} \\
Number of Inserted Bases: {{I}} \\
Number of Deleted Bases: {{D}} \\
Number of Softclipped Bases: {{S}}\\
Number of HardClipped Bases: {{H}}\\
Longest Read: {{longestRead}}\\
Longest Aligned Read : {{longestAlignedRead}}\\


% \begin{verbatim}
% {{refRead}}
% {{alignedRead}}
% \end{verbatim}


\begin{figure}[!htbp]
  \centering
  \includegraphics[width=10cm]{{{histogram_of_read_length}}}
  \caption{histogram of read length}
  \label{fig:histogramofreadlength}
\end{figure}
\begin{figure}[!htbp]
  \centering
  \includegraphics[width=10cm]{{{position_versus_coverage}}}
  \caption{position versus coverage}
  \label{fig:positionversuscoverage}
\end{figure}
\begin{figure}[!htbp]
  \centering
  \includegraphics[width=10cm]{{{read_length_vs_aligned_read_length}}}
  \caption{read length vs aligned read length}
  \label{fig:readlengthvsalignedreadlength}
\end{figure}
	

%%%%%%%%%%%%%%%%%%%%%%%%%%%%%%%%%%%%%%%%%%%%%%%%%%%%%%%%%%%%%%%%%%%%%%
\newpage
\appendix




\end{document}

%%%%%%%%%%%%%%%%%%%%%%%%%%%%%%%%%%%%%%%%%%%%%%%%%%%%%%%%%%%%%%%%%%%%%%
%% The end.
%%%%%%%%%%%%%%%%%%%%%%%%%%%%%%%%%%%%%%%%%%%%%%%%%%%%%%%%%%%%%%%%%%%%%%
